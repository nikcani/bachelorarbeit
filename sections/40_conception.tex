\newpage

\section{Konzeption}

\subsection{Vergleichsmethodik: Principles und Patterns}
Die Analyse hat neun geeignete Principles und acht geeignete Patterns identifiziert.
Die PSR wird, da inhaltlich passend, zusammen mit den Clean Code Regeln separat betrachtet.
So bleiben je acht Principles und Patterns für den Vergleich übrig.

Es ist folgende Methodik angesetzt:
Zuerst ist zu prüfen, ob die Verwendung von Principles und Patterns in den Frameworks dokumentiert ist.
Anschließend ist nach passenden Probleme in der Software zu suchen und zu erkunden, ob diese mit einem passenden Pattern gelöst sind.
Für jedes Pattern wird neben dem Einsatz auch die Qualität und Angemessenheit überprüft.
Anhand des Prototyps ist zu kontrollieren, ob Principles befolgt werden.

Für den abschließenden Vergleich der Frameworks wird der Einsatz von Principles und Patterns primär quantitativ bewertet.
Für jedes befolgte Principle und jedes verwendete Pattern werden zwei Punkte vergeben.
Wenn die Qualität oder Angemessenheit nicht überzeugt, wird nur ein Punkt vergeben.

\subsection{Anforderungen an den neuen Prototyp}
Grundsätzlich soll geprüft werden, ob filament die Probleme beheben kann, die es bei der Umsetzung mit Nova gab.
Allerdings muss auch die Basis überprüft werden, denn es soll verhindert werden, dass Features verloren gehen, die mit Nova gut funktionieren.
Hierfür werden aus den inhaltlichen und technischen Anforderungen im Praxisprojekt die Anforderungen an den neuen Prototypen abgeleitet.
Auf diese Weise wird überprüft, ob filament diese Features ebenfalls umsetzen kann.

\subsubsection{Inhaltliche Anforderungen}
Analog zum bestehenden System soll filament diese inhaltlichen Anforderungen berücksichtigen:
\begin{itemize}
    \item Benutzerverwaltung
    \item Verwaltung von Bodenstationen
    \item Sprachliche Internationalisierung
    \item Anzeige eindimensionaler Performancekriterien je Flug
    \item Anzeige eindimensionaler Performancekriterien je Station
    \item Flugdaten in Echtzeit verfolgbar und im Nachhinein auswertbar
    \item Kartendarstellung von einem oder mehreren Flügen
    \item Messdatentabelle
    \item Messdaten je Flug in zweidimensionalen Liniendiagrammen
    \item Thermodynamische Diagramme
\end{itemize}

\subsubsection{Technische Anforderungen}
Analog zum bestehenden System ist filament hinsichtlich dieser technischen Anforderungen zu prüfen:
\begin{itemize}
    \item Hosting wie bisher, auch On-Premises
    \item Nutzung an einem Desktop Computer
    \item Webbasiert
    \item Responsive UI
\end{itemize}

\subsection{Vergleichsmethodik: Funktionsumfang}
Der Vergleich der Features wird nach folgender Methode durchgeführt:
Jedes Feature wird betrachtet und erhält eine von drei möglichen Bewertungen.
Good (g) wenn das Feature vorhanden ist.
Neutral (n) wenn das Feature vorhanden ist, gegenüber dem anderen Framework jedoch Einschränkungen zeigt.
Bad (b) wenn ein Feature fehlt.
Die Bewertungen werden tabellarisch dargestellt.
Für jedes Good (g) wird ein Punkt addiert und für jedes Bad (b) ein Punkt subtrahiert.
Die Summe der Punkte stellt das quantitative Fazit dar.

Diese Methodik blickt zunächst neutral auf die einzelnen Features, ohne diese zu gewichten.
Insofern kommt die Entscheidung, welche Features betrachtet und welche Bereiche in weitere einzelne Punkte aufgeteilt werden, bereits einer Gewichtung gleich.
