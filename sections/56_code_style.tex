\subsection{Vergleich des Code Style}
\newlineparagraph{PSR}
Nova und filament basieren auf dem Framework Laravel.
Laravel folgt laut eigenen Angaben beim Autoloading der aktuellen PSR-4, beim Coding Style allerdings der veralteten PSR-2~\cite{laravel-docs-coding-style}.
Weder Nova noch filament erwähnen einen Code Style in der jeweiligen Dokumentation.

Filament scheint Laravel Pint~\cite{laravel-docs-pint} zu verwenden, zumindest existiert im Projektrepository eine pint.json Datei mit der notwendigen Konfiguration.
Darin wird als Vorlage der Stil von Laravel verwendet und leicht angepasst.
An dieser Stelle wäre auch die PSR-12 möglich, filament scheint sich aber eher an dem Framework zu orientieren, auf dem es aufbaut.

Bei Nova findet sich im Projektverzeichnis ebenfalls eine pint.json Datei mit der Laravel Vorlage.
Dies ist auch erwartbar, da es sich bei Nova um ein first-party Package handelt.

Zusammenfassend lässt sich also sagen, dass beide Frameworks sich nicht direkt an die aktuelle PSR halten, sondern an die Vorgaben von Laravel selbst.
Die Laravel Standards basieren jedoch zumindest teilweise auf den PSR Standards.

\newlineparagraph{Typisierung}
Bei den zu erweiternden Klassen fallen Unterschiede zwischen den Frameworks auf.
Nova typisiert den Code nur in Teilen.
Vor allem Funktionsparameter sind typisiert, Rückgabewerte nicht immer und Klassenattribute meistens nicht.
Bei filament stellt sich dies anders dar.
In der betrachteten Stichprobe waren die Klassen vollständig typisiert.
Die Qualität des filament Codes ist in diesem Aspekt daher wesentlich höher.

\newlineparagraph{Clean Code}
Die Heuristiken für sauberen Code nach Robert C. Martin sind sehr umfangreich, insgesamt 63, ausgenommen der Java spezifischen.
Im Folgenden wird daher nur ein grober und zusammenfassender Blick auf die Anwendung dieser Heuristiken in den beiden Frameworks geworfen.

Grundsätzlich lässt sich feststellen, dass beide Frameworks sich an die Regeln für Kommentare halten.
Die Umgebung ist ebenfalls einfach einzurichten und erfordert keine komplizierten Schritte, um lauffähig zu werden.
Funktionsargumente sind meist eher wenige und es wird meist sauber mit veränderlichen Objekten gearbeitet.

Die Absicht des Frameworkcodes ist in der Regel klar und deutlich.
Benennungen von Variablen und Funktionen sind beschreibend, eindeutig und enthalten keine unnötigen Präfixe.
Bezüglich der Testabdeckung lässt sich bei beiden Frameworks keine Aussage treffen.

Grundsätzlich lässt sich bei beiden Frameworks kein wirklicher Unterschied bezüglich der \enquote{Sauberkeit} des Codes feststellen.
