\subsection{Vergleich des Code Style}
\newlineparagraph{PSR}
Nova und filament basieren auf dem Framework Laravel.
Laravel folgt laut eigenen Angaben beim Autoloading der aktuellen PSR-4, beim Coding Style allerdings der veralteten PSR-2 (\cite{laravel-docs-coding-style}).
Weder Nova noch filament erwähnen einen Code Style in der jeweiligen Dokumentation.

Filament scheint Laravel Pint (\cite{laravel-docs-pint}) zu verwenden, zumindest existiert im Projektrepository eine pint.json Datei mit der notwendigen Konfiguration.
Darin wird als Vorlage der Stil von Laravel verwendet und leicht angepasst.
An dieser Stelle wäre auch die PSR-12 möglich, filament scheint sich aber eher an dem Framework zu orientieren, auf dem es aufbaut.

Bei Nova findet sich im Projektverzeichnis ebenfalls eine pint.json Datei mit der Laravel Vorlage.
Dies ist auch erwartbar, da es sich bei Nova um ein first-party Package handelt.

Zusammenfassend lässt sich also sagen, dass beide Frameworks sich nicht direkt an die aktuelle PSR halten, sondern an die Vorgaben von Laravel selbst.
Die Laravel Standards basieren jedoch zumindest teilweise auf den PSR Standards.

\newlineparagraph{Typisierung}
\color{red}
tbd
\color{black}

\newlineparagraph{Clean Code}
\color{red}
tbd
\color{black}
