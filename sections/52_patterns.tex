\color{red}

\subsection{Vergleich anhand von Design Patterns}
Die zu vergleichenden Patterns sind bei Nova und filament nicht dokumentiert.
In beiden Dokumentationen ist allerdings eine Möglichkeit für Komponenten zur Kommunikation mittels Events dokumentiert, dies ist vergleichbar mit dem Publisher-Subscriber Pattern.
Alle weiteren Patterns bzw.\ deren Verwendung sind anhand des Prototyps zu prüfen.

\subsubsection{MVC}
Sind passende Probleme mit dem Pattern gelöst?

\subsubsection{PAC}
Sind passende Probleme mit dem Pattern gelöst?

\subsubsection{View Handler}
Sind passende Probleme mit dem Pattern gelöst?

\subsubsection{Whole-Part}
Sind passende Probleme mit dem Pattern gelöst?

\subsubsection{Layers}
Sind passende Probleme mit dem Pattern gelöst?

\subsubsection{Publisher-Subscriber}
Sind passende Probleme mit dem Pattern gelöst?
https://filamentphp.com/docs/2.x/forms/advanced#using-form-events
https://nova.laravel.com/docs/4.0/customization/frontend.html#event-bus

\subsubsection{Broker}
Sind passende Probleme mit dem Pattern gelöst?

\subsubsection{Proxy}
Sind passende Probleme mit dem Pattern gelöst?

\subsubsection{Fazit}
Anhand der Tabelle~\ref{tab:bewertung-patterns} wird deutlich, dass XXX in Bezug auf den Einsatz von Patterns besser zu bewerten ist.

\begin{table}[]
    \caption{Bewertung Patterns}
    \label{tab:bewertung-patterns}
    \centering
    \begin{tabular}{|l|c|c|}
        \hline
        \textbf{Patterns}    & \textbf{Nova} & \textbf{filament} \\ \hline
        MVC                  & X             & X                 \\ \hline
        PAC                  & X             & X                 \\ \hline
        View Handler         & X             & X                 \\ \hline
        Whole-Part           & X             & X                 \\ \hline
        Layers               & X             & X                 \\ \hline
        Publisher-Subscriber & X             & X                 \\ \hline
        Broker               & X             & X                 \\ \hline
        Proxy                & X             & X                 \\ \hline
        \textbf{Summe}       & \textbf{X}    & \textbf{X}        \\ \hline
    \end{tabular}
\end{table}

\color{black}
