\subsection{Rückmeldungen aus der Community}
Im Laravel Subreddit wurde die Frage gestellt, ob Nova oder filament für ein mittelgroßes Admin Panel geeigneter ist\cite{reddit-laravel-nova-vs-filament}.
Dort finden sich die folgenden Antworten auf diese Frage:

\begin{itemize}
    \item \enquote{I have worked with both and found Filament much easier to work with.}
    \item \enquote{Another problem I encountered with Nova is that customers without a technical background have a difficult time to work with Nova.}
    \item \enquote{Customising nova is really difficult.}
    \item \enquote{I've used both, and i still use both , but i prefer filament for it's community and easy customization, but i still use nova for vue projects.}
    \item \enquote{Nova is perfect if you don't need customization}
    \item \enquote{In my experience if you don’t need alot of custom work use Nova cause it’s easier and way faster to develop with.}
    \item \enquote{If you need something custom use filament.}
    \item \enquote{Filament is a lot easier to customize and extend. Mainly because you don’t have the compile Vue, etc.}
\end{itemize}

Insgesamt wird filament als anpassbarer und einfacher in der Handhabung beschrieben.
Durch den TALL Stack (Tailwind, Alpine.js, Laravel \& Livewire) fällt Vue.js im Frontend weg und damit die Notwendigkeit den Code zu kompilieren.
Nova wird bevorzugt, wenn keine Anpassungen benötigt werden, da es als einfaches Admin Panel schneller konfiguriert ist.
