\subsection{Patterns}
Patterns, sogenannte Entwurfsmuster, wurden ursprünglich vom Architekten Christopher Alexander geprägt.
Dieser beschrieb in seinem Buch \enquote{A Pattern Language: Towns, Buildings, Construction}\cite{a-pattern-language} im Jahr 1977 zum ersten Mal Muster, unter dessen Einsatz man immer wiederkehrende Probleme lösen kann.
In der Informatik wurden Entwurfsmuster durch die Veröffentlichung der \enquote{Gang of Four} im Jahr 1994 populärer.
In ihrem Buch \enquote{Design Patterns - Elements of Reusable Object-Oriented Software}\cite{gamma-design-patterns} beschreiben Erich Gamma et al.\ 23 unterschiedliche Entwurfsmuster.
Diese sind eingeteilt in die drei Kategorien Creational, Structural und Behavioral.

1999 ergänzte Martin Fowler in \enquote{Patterns of Enterprise Application Architecture}\cite{patterns-of-enterprise-application-architecture} die Kategorie \enquote{Objektrelationale Abbildung} und dazugehörige Muster.
Gregor Hohpe und Bobby Woolf ergänzten 2003 die Kategorie der Messaging Patterns in ihrem Buch \enquote{Enterprise Integration Patterns}\cite{enterprise-integration-patterns}.

Die Menge der Entwurfsmuster ist so umfangreich, dass eine Wiedergabe aller an dieser Stelle weder von Nutzen noch machbar wäre.
Im Folgenden werden daher nur die Muster eingeführt, die im anschließenden Vergleich verwendet werden.

\subsubsection{GRASP (General Responsibility Assignment Software Patterns)}
Sinnvoll im Projektkontext?

GRASP is a large set of rules about which I could write a separate article.
These are the basic principles that we should follow when creating object design and responsibility assignments.
It consists of: Information Expert, Controller, Creator, High Cohesion, Low Coupling, Pure Fabrication, Polymorphism, Protected Variations, Indirection.
