\subsection{Vergleich auf Basis der Features}
\begin{table}[]
    \caption{Featurevergleich}
    \label{tab:feature-comparison}
    \resizebox{\textwidth}{12.2cm}{%
        \begin{tabular}{llclcl}
            \textbf{Category}                        & \textbf{Feature}       & \textbf{10} & \textbf{Nova}                   & \textbf{34} & \textbf{Filament}                        \\
            \hline
            \multirow{2}{*}{\textbf{Technical}}      & Laravel Support        & n           & \textgreater{}= 8.0             & n           & \textgreater{}= 8.0                      \\
            & PHP Support            & g           & \textgreater{}= 7.3             & n           & \textgreater{}= 8.0                      \\
            \hline
            \multirow{10}{*}{\textbf{Resources}}     & Number of fields       & g           & 38                              & b           & 16                                       \\
            & Computed fields        & g           & yes                             & g           & yes                                      \\
            & Repeater fields        & b           & no                              & g           & supported                                \\
            & Builder fields         & b           & no                              & g           & supported                                \\
            & Table polling          & n           & yes, but bad animation          & g           & yes                                      \\
            & Filters                & g           & supported                       & g           & supported                                \\
            & Lenses                 & g           & supported                       & b           & no                                       \\
            & Actions                & n           & title, fields \& responses      & g           & title, fields, responses \& custom views \\
            & Queued Actions         & g           & supported                       & n           & only custom                              \\
            & Action log             & g           & supported                       & b           & no                                       \\
            \hline
            \multirow{3}{*}{\textbf{Custom views}}   & Custom pages/Tools     & n           & complicated but powerful        & g           & simple and powerful                      \\
            & Custom resource pages  & b           & no                              & g           & yes                                      \\
            & Widgets/Cards          & n           & complicated but powerful        & g           & simple and powerful                      \\
            \hline
            \multirow{12}{*}{\textbf{Search}}        & Searchable Columns     & g           & can be defined                  & g           & can be defined                           \\
            & Title                  & g           & can be defined                  & g           & can be defined                           \\
            & Details/Subtitle       & g           & can be defined                  & g           & can be defined                           \\
            & Url                    & b           & no                              & g           & can be defined                           \\
            & Actions                & b           & no                              & g           & can be defined                           \\
            & Full-Text Indexes      & g           & supported                       & b           & no                                       \\
            & Searching Relations    & g           & supported                       & b           & not clear, probably not supported        \\
            & Searching JSON Data    & g           & supported                       & b           & not clear, probably not supported        \\
            & Limits                 & g           & supported                       & b           & no                                       \\
            & Debounce timing        & g           & supported                       & b           & no                                       \\
            & Global disable         & g           & supported                       & b           & no                                       \\
            & Laravel Scout          & g           & supported                       & n           & can be customized                        \\
            \hline
            \multirow{15}{*}{\textbf{Metrics}}       & Value                  & g           & avg/sum/max/min                 & n           & yes, but no predefined functions         \\
            & Trend/Line             & g           & count/avg/sum/max/min           & n           & yes, but no predefined functions         \\
            & Bar                    & b           & no                              & g           & yes                                      \\
            & Bubble                 & b           & no                              & g           & yes                                      \\
            & Polar Area             & b           & no                              & g           & yes                                      \\
            & Radar                  & b           & no                              & g           & yes                                      \\
            & Scatter                & b           & no                              & g           & yes                                      \\
            & Partition/Pie/Doughnut & g           & avg/sum/max/min                 & n           & yes, but no predefined functions         \\
            & Progress               & g           & count/sum                       & b           & no                                       \\
            & Custom chart control   & b           & no                              & g           & yes, configure the chart.js options      \\
            & Table                  & n           & icon, title, subtitle \& action & g           & multiple columns, filters \& actions     \\
            & Ranges                 & g           & can be defined                  & g           & can be defined                           \\
            & Caching                & g           & supported                       & b           & no                                       \\
            & Formatting             & n           & possible but limited            & g           & possible                                 \\
            & Polling                & n           & some events                     & g           & interval                                 \\
            \hline
            \multirow{3}{*}{\textbf{Dashboard}}      & widget width           & n           & yes                             & g           & yes, responsive                          \\
            & grid col number        & b           & no                              & g           & yes, responsive                          \\
            & conditionally hiding   & g           & yes                             & g           & yes                                      \\
            \hline
            \multirow{5}{*}{\textbf{Notifications}}  & Content                & n           & title, action, icon \& type     & g           & title, body, action, icon \& type        \\
            & Global disable         & g           & supported                       & b           & no                                       \\
            & Duration               & b           & no                              & g           & supported                                \\
            & Persistent             & b           & no                              & g           & supported                                \\
            & Custom view            & b           & no                              & g           & supported                                \\
            \hline
            \multirow{2}{*}{\textbf{Navigation}}     & Customize Main Menu    & g           & yes                             & g           & yes                                      \\
            & Customize User Menu    & g           & yes                             & g           & yes                                      \\
            \hline
            \multirow{10}{*}{\textbf{Customization}} & Logo                   & n           & only SVG file                   & g           & custom view                              \\
            & Dark Mode              & n           & switch can be disabled          & g           & mode and switch can be enabled           \\
            & Collapsible sidebar    & b           & no                              & g           & yes                                      \\
            & Themes                 & n           & colors can be changed           & g           & full theme can be extended               \\
            & Max content width      & b           & no                              & g           & changeable globally or per page          \\
            & Custom Assets          & n           & only by overwriting the layout  & g           & hooks to register styles/scripts         \\
            & Meta tags              & n           & only by overwriting the layout  & g           & yes, via hook                            \\
            & Notification position  & b           & no                              & g           & vertical and horizontal alignment        \\
            & Render hooks           & b           & no                              & g           & add custom views in 28 places            \\
            & Footer                 & n           & only text                       & g           & custom view                              \\
            \hline
            \multirow{3}{*}{\textbf{Plugins}}        & Total                  & n           & 816                             & n           & 158                                      \\
            & Current                & g           & 161                             & g           & 158                                      \\
            & Repository             & n           & third-party\cite{nova-packages} & g           & first-party\cite{filament-plugins}       \\
            \hline
            \multirow{3}{*}{\textbf{Other}}          & Impersonation          & g           & yes                             & b           & no                                       \\
            & Localization           & g           & yes                             & g           & yes                                      \\
            & Modal resources        & b           & no                              & g           & yes
        \end{tabular}%
    }
\end{table}

\newpage

Der Vergleich der Features wir nach folgender Methode durchgeführt.
Jedes Feature wird verglichen und erhält eine von drei möglichen Bewertungen.
Good (g) wenn das Feature vorhanden ist.
Neutral (n) wenn das Feature vorhanden ist, gegenüber dem anderen Framework aber gewisse Einschränkungen mit sich bringt.
Bad (b) wenn ein Feature fehlt.
Die Bewertungen werden tabellarisch\ref{tab:feature-comparison} gelistet.

Das Fazit wird abschließend quantitativ bewertet.
Für jedes g wird ein Punkt addiert und für jedes b ein Punkt subtrahiert.
So ergibt sich für Nova ein Score von 10 und für filament 33.
Der Vergleich der Features ist sehr eindeutig und filament kann hier überzeugen.

Qualitativ überzeugt filament ebenfalls, da insbesondere im Bereich der Customization mehr Features und umfangreichere Optionen vorhanden sind.
Nova überzeugt gegenüber filament im Bereich Search, dieser ist im Projekt Sounding Console allerdings nicht sehr relevant.
