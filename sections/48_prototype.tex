\subsection{Prototypischer Vergleich}
Im Rahmen dieser Arbeit wurde ein Prototyp der Software mit filament entwickelt.
Beide Admin-Panels koexistieren im selben Projekt.
Die neue Variante ist mit dem Präfix \enquote{/admin} in der Route aufrufbar.

Im Gegensatz zum MVP mit Nova in der Praxisprojektarbeit, handelt es sich hier allerdings um einen reinen Prototyp.
Daher liegt der Fokus nicht auf der vollständigen Umsetzung aller Features.
Jedoch ist das bisherige Featureset der Sounding Console im neuen Prototyp nahezu vollständig abgebildet.

Der Fokus liegt insbesondere auf den Bereichen, die mit Nova eher problematisch waren.
So zum Beispiel die Auswertung vergangener Flüge.

\subsubsection{Positive Erfahrungen}
Die Erfahrungen mit filament waren insgesamt sehr positiv.
Vor allem viele Kleinigkeiten sind in filament schöner umgesetzt.
Ein Beispiel ist die Filterung von Tabellen.
Ausgewählte Filter werden oberhalb der Tabelle dargestellt.
Bei Nova versteckt sich dies erst hinter einem Klick in einem Popup.

\newlineparagraph{Flight View Anpassbarkeit}
tbd

\newlineparagraph{Generelle Anpassbarkeit}
tbd

\newlineparagraph{Ladeanimation der aktualisierenden Tabelle}
tbd

\newlineparagraph{Livewire Polling (Flight Dashboard Component)}
tbd

\newlineparagraph{Styling von Custom Components}
tbd

\newlineparagraph{Infinite Loading Tables}
Better Performance with many rows.

Can show ALL rows instead of infinite load.

tbd

\newlineparagraph{Tabellensortierung}
tbd

\subsubsection{Negative Erfahrungen}
Trotz aller positiven Erfahrungen gab es auch eine Hürde mit filament.
Diese konnte allerdings überwunden werden, es musste lediglich auf eine neues Konzept umgestellt werden.

\newlineparagraph{Livewire Polling (Map Component)}
tbd

\subsubsection{Implizit versus explizit}
siehe community: nova ist etwas schneller, felder in filament zu konfigurieren ist expliziter und dadurch aufwändiger

tbd
