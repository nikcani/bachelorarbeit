\subsection{Prototypischer Vergleich}
Im Rahmen dieser Arbeit wurde ein Prototyp der Software mit filament entwickelt.
Beide Admin-Panels koexistieren im selben Projekt.
Die neue Variante ist mit dem Präfix \enquote{/admin} in der Route aufrufbar.

Im Gegensatz zum MVP mit Nova in der Praxisprojektarbeit, handelt es sich hier allerdings um einen reinen Prototyp.
Daher liegt der Fokus nicht auf der vollständigen Umsetzung aller Features.
Jedoch ist das bisherige Featureset der Sounding Console im neuen Prototyp nahezu vollständig abgebildet.

Der Fokus liegt insbesondere auf den Bereichen, die mit Nova eher problematisch waren.
So zum Beispiel die Auswertung vergangener Flüge.

\subsubsection{Positive Erfahrungen}
Die Erfahrungen mit filament waren insgesamt sehr positiv.
Vor allem viele Kleinigkeiten sind in filament schöner umgesetzt.
Ein Beispiel ist die Filterung von Tabellen.
Ausgewählte Filter werden oberhalb der Tabelle dargestellt.
Bei Nova versteckt sich dies erst hinter einem Klick in einem Popup.

\newlineparagraph{Flight View Anpassbarkeit}
Manche Ansichten benötigen weitreichende Anpassungen, so vor allem die Ansicht eines vergangenen Fluges.
Bei laufenden Flügen wird lediglich ein Dashboard mit aktuellen Werten und eine Karte angezeigt.
Beides passt sinnvoll oberhalb der restlichen Komponenten der Ressource, den Stammdatenfeldern und der Messwertetabelle.

Im Fall von vergangenen Flügen fällt das Dashboard weg, jedoch kommen ein Messdatendiagramm, eine Statistikübersicht und ein Skew-T-Diagramm hinzu.
Insgesamt zu viele und zu große Komponenten, um diese direkt oberhalb auf der Seite zu platzieren.
Ein User müsste zu viel scrollen, um alles zu erfassen, dies wäre unübersichtlich und ineffizient.

Nova bietet kaum Möglichkeiten eine Ansicht anzupassen.
Daher war die beste Variante, die Komponenten in Unterpunkte auszulagern.
Nova unterstützt sogenannte Tools, welche eigenständige Seiten sind, auf denen individuelle Komponenten implementiert werden können.
Außerdem kann kontextabhängig das Menü angepasst werden.
Tests erster Nutzer innerhalb des Teams zeigten allerdings, dass die entworfene Navigationsstruktur zu unübersichtlich ist.

Filament bietet gegenüber Nova weitaus mehr Möglichkeiten, eine Ansicht individuell anzupassen.
So kann bei der vorliegenden Herausforderung eine individuelle View für den Header angegeben werden.
Diese wird oberhalb aller anderen Komponenten auf der Ansichtsseite eines Fluges gerendert.
Innerhalb dieser individuellen View kann dann eine Tab View implementiert werden, die immer nur eine größere Komponente auf einmal anzeigt.

Die Lösung mit filament überzeugt durch eine deutlich bessere User Experience.
Die komplizierte Navigationsstruktur im Menü entfällt komplett und alles findet direkt innerhalb der Flugansicht statt.
Längeres Scrollen entfällt und die Ansicht bleibt kompakt und übersichtlich.
Filament überzeugt durch die flexible und umfangreiche Anpassbarkeit.

\newlineparagraph{Generelle Anpassbarkeit}
tbd

\newlineparagraph{Ladeanimation der aktualisierenden Tabelle}
tbd

\newlineparagraph{Livewire Polling (Flight Dashboard Component)}
tbd

\newlineparagraph{Styling von Custom Components}
tbd

\newlineparagraph{Infinite Loading Tables}
Better Performance with many rows.

Can show ALL rows instead of infinite load.

tbd

\newlineparagraph{Tabellensortierung}
tbd

\subsubsection{Negative Erfahrungen}
Trotz aller positiven Erfahrungen gab es auch eine Hürde mit filament.
Diese konnte allerdings überwunden werden, es musste lediglich auf eine neues Konzept umgestellt werden.

\newlineparagraph{Livewire Polling (Map Component)}
tbd

\subsubsection{Implizit versus explizit}
siehe community: nova ist etwas schneller, felder in filament zu konfigurieren ist expliziter und dadurch aufwändiger

tbd
