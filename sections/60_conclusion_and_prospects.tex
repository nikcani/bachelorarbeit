\newpage

\section{Fazit und Ausblick}

\subsection{Fazit}
\color{red}
tbd
\color{black}

\subsection{Ausblick}
Aufgrund des eindeutigen Fazits zugunsten des Einsatzes von filament soll die Sounding Console zeitnah vollständig auf filament umgestellt werden.
Dafür müssen die wenigen, gegenüber dem MVC mit Nova fehlenden, Features implementiert werden.

Im Anschluss daran soll eine erste interne Testphase starten und die Software von mehreren Nutzern evaluiert werden.
Danach soll die Software bei den ersten Messnetzwerken in Betrieb gehen.

Bisher wurden realistische Lasttests noch nicht durchgeführt.
Diese wurden bereits im Ausblick der Praxisprojektarbeit angekündigt und die Bearbeitung ist bereits gestartet.
Jedoch wurde zunächst der Vergleich mit, sowie der Umbau auf filament priorisiert.
Da bei den Lasttests kein Browser ausgeführt wird, sondern nur HTTP Endpunkte abgerufen werden, ist das Testing von filament voraussichtlich besser zu bewerkstelligen.
Nova setzt viel auf Code im Frontend und führt auch dort das Routing durch.
Filament setzt die meisten Funktionen rein im Backend um und lässt sich insofern mit der angestrebten Testmethode auch besser testen, als dies bei Nova der Fall wäre.

Sobald die Software sich auch im produktiven Betrieb bewiesen hat, sollen weitere Funktionen, zum Beispiel für automatisierte Bodenstationen, sogenannte Autolauncher, implementiert werden.
Hierzu liegen derzeit jedoch nur grobe Pläne vor.
