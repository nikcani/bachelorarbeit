\section{Einleitung}

Diese Arbeit baut auf einer vorangegangenen Praxisprojektarbeit auf.
Sie trägt den Titel \enquote{Praktische Evaluation des Frameworks Laravel Nova am Beispiel einer Anwendung zur Verwaltung und Auswertung von Radiosondenaufstiegen}.
Als Ergebnis der Praxisprojektarbeit wurde die Sounding Console entwickelt.
Dabei zeigten sich Probleme mit Nova, insbesondere in puncto Anpassbarkeit.
Gleichwohl wurden im Rahmen der Nutzwertanalyse des Projekts geeignete Alternativen zu Nova identifiziert.
Dabei wurde vor allem das Framework filament~\cite{filament} als bessere Möglichkeit in Betracht gezogen.

Die Bachelorarbeit setzt auf den Ergebnissen des genannten Praxisprojekts auf.
Die beiden Frameworks Nova und filament sollen anhand von Software Design Principles und Patterns, sowie prototypischer Implementierung, verglichen werden.

Der Verfasser dieser Arbeit ist für die \enquote{GRAW Radiosondes GmbH \& Co. KG}\cite{graw} tätig und bearbeitet dort als Werkstudent die vorliegende Arbeit.

\subsection{Fragestellung}
Im Rahmen dieser Arbeit soll die Frage geklärt werden, ob für das vorliegende Projekt der \enquote{Sounding Console} Nova oder filament besser geeignet ist.
Dabei steht die Problematik bezüglich der Anpassbarkeit im Vordergrund.

\subsection{Zielsetzung}
Ziel dieser Arbeit ist zunächst ein theoretischer Vergleich der beiden Frameworks Nova und filament.

Vor diesem Hintergrund sollen anhand prototypischer Umsetzungen einiger Features der Sounding Console mittels filament praktische Erfahrungen mit dem Framework gesammelt werden.
Diese sollen mit bereits bestehenden Erkenntnissen mit Nova verglichen werden.

Im Ergebnis soll abschließend die oben genannte Fragestellung beantwortet und auf dieser Grundlage der weitere Weg für die Applikation bestimmt werden.
