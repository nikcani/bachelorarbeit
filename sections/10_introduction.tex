\section{Einleitung}

Diese Arbeit baut auf der vorangegangenen Praxisprojektarbeit auf.
Diese trägt den Titel \enquote{Praktische Evaluation des Frameworks Laravel Nova am Beispiel einer Anwendung zur Verwaltung und Auswertung von Radiosondenaufstiegen}.
Als Ergebnis der Praxisprojektarbeit wurde die Sounding Console entwickelt.
Dabei zeigten sich Probleme mit Nova, vor allem in puncto Anpassbarkeit.
Ebenfalls wurden in einer Nutzwertanalyse geeignete Alternativen zu Nova identifiziert.
Dabei wurde vor allem das Framework filament~\cite{filament} als bessere Alternative herausgearbeitet.

Unter Betrachtung dieser kurzen Vorstudie entsteht die Bachelorarbeit.
Die beiden Frameworks Nova und filament sollen anhand von Software Design Principles und Patterns, sowie prototypischer Implementierung, verglichen werden.

Der Verfasser dieser Arbeit ist mittlerweile direkt für die \enquote{GRAW Radiosondes GmbH \& Co. KG}\cite{graw} tätig und bearbeitet dort die vorliegende Arbeit.

\subsection{Fragestellung}
Im Rahmen dieser Arbeit soll die Frage geklärt werden, ob für das vorliegende Projekt der \enquote{Sounding Console} Nova oder filament besser geeignet ist.
Insbesondere die Probleme bezüglich der Anpassbarkeit sollen verglichen werden.

\subsection{Zielsetzung}
Ziel dieser Arbeit ist zum einen die prototypische Umsetzung einiger Features der Sounding Console mittels filament.
Dabei sollen praktische Erfahrungen mit dem Framework gesammelt werden, um diese mit den bei Nova vorhandenen Erfahrungen zu vergleichen.

Zum anderen soll ein theoretischer Vergleich der beiden Frameworks durchgeführt werden.

Als finales Ergebnis soll die Fragestellung bewertet werden und damit der weitere Weg für die Applikation bestimmt werden.
