\addtocontents{toc}{\protect\newpage}

\newpage


\section{Ergebnisse und Reflektion}

\subsection{Ergebnisse}
Beim Einsatz von Design Principles liegt filament vor Nova, bei Design Patterns sind beide gleich auf.
Durch die prototypische Implementierung mit filament hat sich gezeigt, dass dies viele Vorteile mit sich bringt.
Der Prototyp ist in seiner gesamten UX (User Experience) besser als das MVP auf Basis von Nova.
Daher ist in diesem Aspekt filament eindeutig besser.

Auch beim Vergleich des Funktionsumfangs kann filament klar überzeugen.
Die Maturity ist ausgeglichen, hervorzuheben sind jedoch die Pläne in der Roadmap von filament, die für das Projekt Sounding Console sehr vorteilhaft sind.
In einigen Jahren wird filament bei der Maturity daher sogar wahrscheinlich besser eingeordnet als Nova.
Obwohl filament über eine stärkere Typisierung verfügt, genügt dies nicht für einen klaren Unterschied beim Code Style.
Beide Code Styles sind insgesamt ähnlich gut.

Bezüglich Kosten und Lizenzierung kann erneut filament überzeugen.
Die Meinungen aus der Community sind nicht unbedingt wertend, im Projektkontext stellt sich dies allerdings anders dar.
Insbesondere die von mehreren Personen hervorgehobene bessere Anpassbarkeit von filament konnte bei der Sounding Console einige Probleme lösen.
Auch für die Zukunft und damit für weitere Anforderungen ist es definitiv von Vorteil, vielseitige Anpassungen vornehmen zu können.
Daher geht auch dieser Punkt an filament.

Die Vergleichsaspekte sind in der Tabelle~\ref{tab:vergleichsergebnisse} ersichtlich und summiert.
Mit 8 zu 3 Punkten fällt das Vergleichsergebnis klar zugunsten von filament aus.

\begin{table}[h!]
    \centering
    \caption{Vergleichsergebnisse}
    \label{tab:vergleichsergebnisse}
    \begin{tabular}{|l|c|c|}
        \hline
        \textbf{Vergleichsaspekt} & \textbf{Nova} & \textbf{filament} \\ \hline
        Design Principles         & 0             & 1                 \\ \hline
        Design Patterns           & 1             & 1                 \\ \hline
        Prototyp                  & 0             & 1                 \\ \hline
        Funktionsumfang           & 0             & 1                 \\ \hline
        Maturity                  & 1             & 1                 \\ \hline
        Code Style                & 1             & 1                 \\ \hline
        Kosten/Lizenzierung       & 0             & 1                 \\ \hline
        Community                 & 0             & 1                 \\ \hline
        \textbf{Summe}            & \textbf{3}    & \textbf{8}        \\ \hline
    \end{tabular}
\end{table}

\newpage

\subsection{Kritischer Rückblick}
Der theoretische Vergleich stelle sich komplizierter dar, als ursprünglich vermutet.
Insbesondere anwendbare Patterns zu finden ist trotz der vielzahl der verfügbaren Patterns nicht trivial.
Viele Patterns liegen in anderen Ebenen der Software.
Dies ist auch teilweise erst im Vergleich aufgefallen.

Die Anwendung der Principles verlief gut.
Auch die gewonnenen Erfahrungen aus der prototypischen Entwicklung waren umfangreich und aussagekräftig.
Ebenso auch der theoretische Funktionsvergleich.
Auch in den anderen Vergleichskategorien konnten Erkenntnisse gewonnen werden.

Insgesamt lässt sich eine fundierte Aussage treffen, welches Framework im Projektkontext geeigneter ist.
Auch für andere Projekte ist filament vermutlich eine bessere Wahl.
In Zukunft wird sich zeigen, wie sich beide Frameworks weiterentwickeln.
Aktuell ist nicht zu erwarten, dass sich dadurch etwas an diesem Vergleichsergebnis ändert, jedoch muss dies immer wieder validiert werden.

Sollte sich in Zukunft zeigen, dass auch filament bezüglich der Anpassbarkeit an Grenzen stößt und mit funktionalen Anforderungen an die Anwendung kollidiert, dann wäre der einzige mögliche Schritt eine vollständige individuelle Entwicklung.
Aktuell scheint es kein Framework zu geben, das besser geeignet für eine individuelle Anwendung im Stile eines Admin Panels ist.
Eine komplette Individualentwicklung hätte zwar Vorteile hinsichtlich der flexibilität, würde aber auch den Entwicklungsaufwand stark erhöhen.
Daher ist filament zurzeit definitiv die bessere Wahl.
