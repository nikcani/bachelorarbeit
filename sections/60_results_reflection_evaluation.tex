\addtocontents{toc}{\protect\newpage}

\newpage


\section{Ergebnisse und Reflektion}

\subsection{Ergebnisse}
Beim Einsatz von Design Principles liegt filament vor Nova, bei Design Patterns sind beide gleich auf.
Durch die prototypische Implementierung mit filament hat sich gezeigt, dass dies zahlreiche Vorteile mit sich bringt.
Der Prototyp ist in seiner gesamten UX (User Experience) besser als das MVP auf Basis von Nova.
Aus diesem Blickwinkel heraus ist eindeutig filament zu bevorzugen.

Weiterhin kann filament beim Vergleich des Funktionsumfangs klar überzeugen.
Die Maturity ist ausgeglichen und besonders hervorzuheben sind die Pläne in der Roadmap von filament, die für das Projekt Sounding Console sehr vorteilhaft sind.
Mit Blick auf diese Aspekte ist davon auszugehen, dass filament in einigen Jahren bei der Maturity voraussichtlich besser einzuordnen ist als Nova.
Obwohl filament über eine stärkere Typisierung verfügt, genügt dies nicht für einen klaren Unterschied beim Code Style.
Beide Code Styles sind insgesamt ähnlich gut.

Bezüglich Kosten und Lizenzierung kann erneut filament überzeugen.
Die Meinungen aus der Community sind nicht unbedingt wertend, im konkreten Projektkontext stellt sich dies anders dar:
Insbesondere die von mehreren Personen hervorgehobene bessere Anpassbarkeit von filament konnte bei der Sounding Console einige Probleme lösen.
Auch für weitere zukünftige Anforderungen ist es von Vorteil, vielseitige Anpassungen vornehmen zu können.
Insofern geht auch dieser Punkt an filament.

Die Vergleichsaspekte sind in der Tabelle~\ref{tab:vergleichsergebnisse} ersichtlich und summiert.
Mit 8 zu 3 Punkten fällt das Ergebnis eindeutig zugunsten von filament aus.

\begin{table}[h!]
    \centering
    \caption{Vergleichsergebnisse}
    \label{tab:vergleichsergebnisse}
    \begin{tabular}{|l|c|c|}
        \hline
        \textbf{Vergleichsaspekt} & \textbf{Nova} & \textbf{filament} \\ \hline
        Design Principles         & 0             & 1                 \\ \hline
        Design Patterns           & 1             & 1                 \\ \hline
        Prototyp                  & 0             & 1                 \\ \hline
        Funktionsumfang           & 0             & 1                 \\ \hline
        Maturity                  & 1             & 1                 \\ \hline
        Code Style                & 1             & 1                 \\ \hline
        Kosten/Lizenzierung       & 0             & 1                 \\ \hline
        Community                 & 0             & 1                 \\ \hline
        \textbf{Summe}            & \textbf{3}    & \textbf{8}        \\ \hline
    \end{tabular}
\end{table}

\newpage

\subsection{Kritischer Rückblick}
Der theoretische Vergleich stellte sich komplex dar.
Obwohl zahlreiche Patterns zu finden sind, zeigte sich in der näheren Analyse, dass sie in der Anwendung keinen unmittelbaren Mehrwert bieten.
Vielmehr wird im Vergleich offensichtlich, dass viele Patterns in anderen Ebenen der Software liegen.

Die Anwendung der Principles verlief gut.
Auch die gewonnenen Erfahrungen aus der prototypischen Entwicklung waren umfangreich und aussagekräftig.
Ähnlich verlief der theoretische Funktionsvergleich.
Ebenso konnten in den anderen Vergleichskategorien hilfreiche Erkenntnisse gewonnen werden.

Insgesamt lässt sich auf dieser Basis eine fundierte Aussage treffen, welches Framework im Projektkontext geeigneter ist.
Auch für andere Projekte ist filament vermutlich eine bessere Wahl, da vor allem die Anpassbarkeit wesentlich besser ist.
Dies ist allerdings für jedes Projekt im Einzelfall zu prüfen.
Wie sich die beiden Frameworks weiterentwickeln, lässt sich für die Zukunft nicht vorhersehen.
Derzeit lässt jedoch nichts darauf schließen, dass sich an diesem dargestellten Vergleichsergebnis etwas ändert, wenngleich es regelmäßig validiert werden muss.

Sollte sich beispielsweise in Zukunft zeigen, dass auch filament bezüglich der Anpassbarkeit an Grenzen stößt und mit funktionalen Anforderungen an die Anwendung kollidiert, dann wäre der einzige mögliche Schritt eine vollständige individuelle Entwicklung.
Aktuell ist es jedoch kein Framework identifiziert, das für eine individuelle Anwendung im Stile eines Admin Panels besser geeignet ist.
Aktuell scheint es kein Framework zu geben, das besser geeignet für eine individuelle Anwendung im Stile eines Admin Panels ist.
Eine komplette Individualentwicklung hätte zwar Vorteile hinsichtlich der Flexibilität, würde aber auch den Entwicklungsaufwand stark erhöhen.
Insofern ist filament derzeit definitiv die bessere Wahl.
