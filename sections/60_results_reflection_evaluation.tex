\addtocontents{toc}{\protect\newpage}

\newpage


\section{Ergebnisse und Reflektion}

\subsection{Ergebnisse}
Beim Einsatz von Design Principles liegt filament vor Nova, bei Design Patterns sind beide gleich auf.
Durch die prototypische Implementierung mit filament hat sich gezeigt, dass dies viele Vorteile mit sich bringt.
Der Prototyp ist in seiner gesamten UX (User Experience) besser als das MVP auf Basis von Nova.
Daher ist in diesem Aspekt filament eindeutig besser.

Auch beim Vergleich des Funktionsumfangs kann filament klar überzeugen.
Die Maturity ist ausgeglichen, hervorzuheben sind jedoch die Pläne in der Roadmap von filament, die für das Projekt Sounding Console sehr vorteilhaft sind.
In einigen Jahren wird filament bei der Maturity daher sogar wahrscheinlich besser eingeordnet als Nova.
Obwohl filament über eine stärkere Typisierung verfügt, genügt dies nicht für einen klaren Unterschied beim Code Style.
Beide Code Styles sind insgesamt ähnlich gut.

Bezüglich Kosten und Lizenzierung kann erneut filament überzeugen.
Die Meinungen aus der Community sind nicht unbedingt wertend, im Projektkontext stellt sich dies allerdings anders dar.
Insbesondere die von mehreren Personen hervorgehobene bessere Anpassbarkeit von filament konnte bei der Sounding Console einige Probleme lösen.
Auch für die Zukunft und damit für weitere Anforderungen ist es definitiv von Vorteil, vielseitige Anpassungen vornehmen zu können.
Daher geht auch dieser Punkt an filament.

Die Vergleichsaspekte sind in der Tabelle~\ref{tab:vergleichsergebnisse} ersichtlich und summiert.
Mit 8 zu 3 Punkten fällt das Vergleichsergebnis klar zugunsten von filament aus.

\begin{figure}[h!]
    \begin{table}[]
        \centering
        \caption{Vergleichsergebnisse}
        \label{tab:vergleichsergebnisse}
        \begin{tabular}{|l|c|c|}
            \hline
            \textbf{Vergleichsaspekt} & \textbf{Nova} & \textbf{filament} \\ \hline
            Design Principles         & 0             & 1                 \\ \hline
            Design Patterns           & 1             & 1                 \\ \hline
            Prototyp                  & 0             & 1                 \\ \hline
            Funktionsumfang           & 0             & 1                 \\ \hline
            Maturity                  & 1             & 1                 \\ \hline
            Code Style                & 1             & 1                 \\ \hline
            Kosten/Lizenzierung       & 0             & 1                 \\ \hline
            Community                 & 0             & 1                 \\ \hline
            \textbf{Summe}            & \textbf{3}    & \textbf{8}        \\ \hline
        \end{tabular}
    \end{table}
\end{figure}

\subsection{Kritischer Rückblick}

