\section{Konzeption}

\subsection{Anforderungen an den neuen Prototyp}

Grundsätzlich soll geprüft werden, ob filament die Probleme beheben kann, die es bei der Umsetzung mit Nova gab.
Allerdings muss auch die Basis überprüft werden.
Daher werden aus den inhaltlichen und technischen Anforderungen im Praxisprojekt nun die Anforderungen an den neuen Prototypen abgeleitet.
So wird auch überprüft, ob filament die Features umsetzen kann, die mit Nova bereits gut funktionieren.

\subsubsection{Inhaltliche Anforderungen (analog zum bestehenden System)}
\begin{itemize}
    \item Benutzerverwaltung
    \item Verwaltung von Bodenstationen
    \item Sprachliche Internationalisierung
    \item Anzeige eindimensionaler Performancekriterien je Flug
    \item Anzeige eindimensionaler Performancekriterien je Station
    \item Flugdaten in Echtzeit verfolgbar und im Nachhinein auswertbar
    \item Kartendarstellung von einem oder mehreren Flügen
    \item Messdatentabelle
    \item Messdaten je Flug in zweidimensionalen Liniendiagrammen gegenüber Zeit, Höhe und Luftdruck
    \item Thermodynamische Diagramme
\end{itemize}

\subsubsection{Technische Anforderungen (analog zum bestehenden System)}
\begin{itemize}
    \item Hosting wie bisher, auch On-Premises
    \item Nutzung an einem Desktop Computer
    \item Webbasiert
    \item UI responsive
\end{itemize}

\newpage

\subsection{Vergleichsmethodik: Funktionsumfang}
Der Vergleich der Features wir nach folgender Methode durchgeführt.
Jedes Feature wird verglichen und erhält eine von drei möglichen Bewertungen.
Good (g) wenn das Feature vorhanden ist.
Neutral (n) wenn das Feature vorhanden ist, gegenüber dem anderen Framework aber gewisse Einschränkungen mit sich bringt.
Bad (b) wenn ein Feature fehlt.
Die Bewertungen werden tabellarisch gelistet.
Für jedes g wird ein Punkt addiert und für jedes b ein Punkt subtrahiert.
Die Summe der Punkte stellt das quantitative Fazit dar.

Diese Methodik ist relativ neutral und verzichtet auf eine Gewichtung der einzelnen Vergleichspunkte.
Grundsätzlich ist dies von Vorteil, allerdings auch nicht perfekt.
So ist die Entscheidung, welche Vergleichspunkte definiert werden und welche Bereiche in weitere einzelne Punkte aufgeteilt werden, letztlich mit einer Gewichtung zu vergleichen.

\subsection{Vergleichsmethodik: Principles und Patterns}
\color{red}
Wie werden die passenden Principles und Patterns verglichen?

tbd
\color{black}
