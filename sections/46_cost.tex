\subsection{Kostenvergleich/Lizenzierung}
Die Kosten beim Einsatz von Nova und filament unterscheiden sich deutlich.
Filament überzeugt durch quelloffene Entwicklung unter der MIT-Lizenz.
Dadurch ist die vollständige Nutzung, Veränderungen und der Vertrieb möglich.
Es entstehen keine Kosten und grundsätzlich wäre auch eine Weiterentwicklung durch neue Entwickler möglich, wenn die aktuellen Entwickler das Projekt einstellen sollten.

Nova hingegen kostet für unlimitierte Projekte einmalig 299 Dollar.
Durch die On-Premise Bereitstellung ist diese größere Lizenz notwendig.
Diese Lizenz ermöglicht Updates für ein Jahr, danach fallen erneut Kosten an.
Da Updates notwendig sind, muss mit jährlich wiederkehrenden Kosten gerechnet werden.
Verglichen mit den Personalkosten sind diese allerdings sehr gering und eher zu vernachlässigen.

Aus dem Kostenvergleich geht filament als klarer Sieger hervor.
Neben den geringeren Kosten ist vor allem die quelloffene Entwicklung von Vorteil, vor allem für lange Produktlebenszyklen.
