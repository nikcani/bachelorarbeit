\newpage

\subsection{Clean Code}
Robert C. Martin nennt in seinem Buch \enquote{Clean Code: A Handbook of Agile Software Craftmanship}\cite{clean-code} verschiedene Heuristiken für \enquote{Clean Code}.
Diese sind aufgeteilt in die folgenden Gruppen:

\newlineparagraph{Kommentare}
In Kommentaren sollten sich nur geeignete Inhalte befinden.
Redundanzen oder veraltete Informationen stören, außerdem sollten Kommentare präzise und sprachlich sauber geschrieben werden.
Auskommentierten Code gilt es unbedingt zu entfernen.

\newlineparagraph{Umgebung}
Ein Projekt sollte in einem trivialen Schritt gebaut werden können, dies gilt ebenfalls für alle Tests.

\newlineparagraph{Funktionen}
Funktionen sollten möglichst wenige Argumente haben, maximal drei.
Anstelle von Rückgabewerten sollte das Objekt verändert werden, auf dem Funktionen ausgeführt werden.
\enquote{Flag Arguments} sind zu ersetzen.
Weiterhin sollten die unterschiedlichen Funktionen aufgeteilt werden.
Methoden, die nicht mehr verwendet werden, gilt es zu entfernen.

\newlineparagraph{Generelles}
Quelldateien sollten möglichst nur eine Sprache enthalten.
Offensichtliche Erwartungen sind zu erfüllen, auch bei Randfällen.
Sicherheitsmechanismen, wie zum Beispiel Tests, sollten nicht überschrieben werden.
Duplikate sind unbedingt zu vermeiden.
Jeder Bestandteil gehört in eine bestimmte Abstraktionsschicht.
Basisklassen sollten deshalb nicht von Ihren Ableitungen abhängen.

Zu viele Informationen gilt es aufzuteilen und zu vereinfachen.
Nicht benötigter Code ist zu löschen.
Funktionen und Variablen sollten nah bei ihrer Verwendung definiert werden.
Inkonsistenzen, Unordnung und künstliche Abhängigkeiten gilt es zu vermeiden.
Features sind an der Stelle zu implementieren, wo sie sinnvoll hingehören, beispielsweise durch die Abhängigkeit von Variablen.
Selektoren in Funktionsargumenten sollten nicht verwendet werden, stattdessen können die Funktionen meist aufgeteilt werden.
Die Absicht des Codes sollte nie versteckt werden.

Methoden gilt es nur statisch zu implementieren, wenn dies Sinn ergibt.
Variablen sollten erklärende Namen tragen, ebenso sollten Funktionen nach ihrem Zweck benannt werden.
Verwendete Algorithmen sollten für den Entwickler einfach nachvollziehbar sein.
Polymorphismen sind komplexen Verzweigungen vorzuziehen.
Standardisierte Konventionen sollten befolgt werden.
Magische Werte sollten nur in benannten Konstanten bestehen.

Code sollte nicht willkürlich, sondern präzise sein und Struktur immer der Konvention voranstellen.
Komplexere Konditionalitäten sollten in Funktionen extrahiert werden, ebenso ist auf negierte Bedingungen zu verzichten.
Funktionen sollten nur eine Sache erledigen und nur eine Abstraktionsebene bearbeiten.
Zeitliche Abhängigkeiten sollten nicht versteckt, sondern im Code vorgegeben sein.
Randbedingungen gilt es explizit zu benennen.
Konfigurierbare Daten sollten in den oberen Ebenen verankert sein und Module nur mit ihren direkten Abhängigkeiten interagieren.

\newlineparagraph{Bennenungen}
Namen sollten stets beschreibend sein und die passende Abstraktionsebene reflektieren.
Möglichst ist dabei eine standardisierte Nomenklatur zu verwenden.
Zweideutigkeiten gilt es zu vermeiden.
Für größere Bereiche sollten auch längere Namen verwendet werden.
Kodierte Präfixe sollten vermieden und Nebeneffekte beschrieben werden.

\newlineparagraph{Tests}
Eine ausreichende Menge Tests ist anzustreben, Coverage Tools helfen bei der Prüfung.
Ebenso sind triviale Tests wichtig.
Tests sollten nicht ignoriert werden.
Randbedingungen und Bereiche um Fehler herum sollten intensiv getestet werden.
Patterns in der Art und Weise, wie Tests fehlschlagen, können das Problem eingrenzen.
Ebenso nützlich sind Test Coverage Patterns.
Tests sollten so schnell wie möglich sein, da sie sonst oft nicht ausgeführt werden.

\subsubsection{The Boy Scout Rule}
Wichtig zu betonen ist, dass Code nicht nur einmal gut geschrieben werden muss.
Code muss auch über die Zeit sauber gehalten werden, um bei Änderungen nicht schlechter zu werden.
In diesem Kontext ist die Regel der amerikanischen Pfadfinder hilfreich:
\enquote{Leave the campground cleaner than you found it.}~\cite{clean-code}
