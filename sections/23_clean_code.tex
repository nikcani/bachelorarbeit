\newpage

\subsection{Clean Code}
Robert C. Martin nennt in seinem Buch \enquote{Clean Code: A Handbook of Agile Software Craftmanship}\cite{clean-code} verschiedene Heuristiken für \enquote{Clean Code}.
Diese sind aufgeteilt in die folgenden Gruppen:

\newlineparagraph{Kommentare}
In Kommentaren sollten sich nur geeignete Inhalte befinden.
Redundanzen oder veraltete Informationen stören, außerdem sollten Kommentare präzise und sprachlich sauber geschrieben werden.
Auskommentierter Code sollte unbedingt entfernt werden.

\newlineparagraph{Umgebung}
Ein Projekt sollte in einem trivialen Schritt gebaut werden können, dies gilt ebenfalls für alle Tests.

\newlineparagraph{Funktionen}
Funktionen sollten möglichst wenige Argumente haben, maximal jedoch drei.
Anstelle von Rückgabewerten sollte das Objekt verändert werden, auf dem Funktionen ausgeführt werden.
\enquote{Flag Arguments} sollten ersetzt werden, die unterschiedlichen Funktionen aufgeteilt werden.
Methoden sollten entfernt werden, wenn sie nicht mehr verwendet werden.

\newlineparagraph{Generelles}
Quelldateien sollten möglichst nur eine Sprache enthalten.
Offensichtliche Erwartungen sollten erfüllt werden, auch bei Randfällen.
Sicherheitsmechanismen, wie zum Beispiel Tests, sollten nicht überschrieben werden.
Duplikate sind unbedingt zu vermeiden.
Jeder Bestandteil gehört in eine bestimmte Abstraktionsschicht.
Basisklassen sollten nicht von Ihren Ableitungen abhängen.

Zu viele Informationen sollten aufgeteilt und vereinfacht, toter Code gelöscht werden.
Funktionen und Variablen sollten nah bei ihrer Verwendung definiert werden.
Inkonsistenzen, Unordnung und künstliche Abhängigkeiten sollten vermieden werden.
Features sollten an der Stelle implementiert werden, wo sie sinnvoll hingehören, beispielsweise durch die Abhängigkeit von Variablen.
Selektoren in Funktionsargumenten sollten nicht verwendet werden, stattdessen können die Funktionen meist aufgeteilt werden.
Die Absicht des Codes sollte nie versteckt werden.

Methoden sollten nur statisch implementiert werden, wenn dies Sinn ergibt.
Variablen sollten erklärende Namen tragen, ebenso sollten Funktionen nach ihrem Zweck benannt werden.
Verwendete Algorithmen sollten dem Entwickler klar sein.
Polymorphismen sind komplexen Verzweigungen vorzuziehen.
Standardisierte Konventionen sollten befolgt werden.
Magische Werte sollten nur in benannten Konstanten bestehen.

Code sollte nicht willkürlich, sondern präzise sein und Struktur immer der Konvention voranstellen.
Komplexere konditionalitäten sollten in Funktionen extrahiert werden, ebenso sollte auf negierte Bedingungen verzichtet werden.
Funktionen sollten nur eine Sache erledigen und nur eine Abstraktionsebene bearbeiten.
Zeitliche abhängigkeiten sollten nicht versteckt sein, sondern im Code vorgegeben sein.
Randbedingungen sollten explizit benannt werden.
Konfigurierbare Daten sollten in den oberen Ebenen verankert sein.
Module sollten nur mit ihren direkten Abhängigkeiten interagieren.

\newlineparagraph{Bennenungen}
Namen sollten stets beschreibend sein und die passende Abstraktionsebene reflektieren.
Standardisierte Nomenklatur sollte möglichst verwendet werden.
Zweideutigkeiten sollten vermieden werden.
Für größere Bereiche sollten auch längere Namen verwendet werden.
Kodierte Präfixe sollten vermieden, Nebeneffekte sollten beschrieben werden.

\newlineparagraph{Tests}
Eine ausreichende Menge Tests ist anzustreben, Coverage Tools helfen bei der Prüfung.
Auch triviale Tests sind wichtig.
Tests sollten nicht ignoriert werden.
Randbedingungen und Bereiche um Fehler herum sollten besonders getestet werden.
Patterns in der Art und Weise, wie Tests fehlschlagen, können das Problem eingrenzen.
Ebenso interessant sind Test Coverage Patterns.
Tests sollten so schnell wie möglich sein, da sie sonst nicht ausgeführt werden.

\subsubsection{The Boy Scout Rule}
Wichtig zu erwähnen bleibt, dass Code nicht nur einmal gut geschrieben werden muss.
Code muss mit der Zeit sauber gehalten werden, um durch Änderungen nicht schlechter zu werden.
In diesem Kontext sollte die Regel der amerikanischen Pfadfinder angewendet werden:
\enquote{Leave the campground cleaner than you found it.}~\cite{clean-code}
