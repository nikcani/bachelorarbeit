\subsection{Principles}
Software Design Principles sind weder Regeln, Gesetze noch perfekte Wahrheiten.
Stattdessen sind diese Prinzipien Empfehlungen bzw.\ Ratschläge.
(Vgl.~\cite{getting-a-solid-start})

Software Design Prinzipien sind Heuristiken und allgemeingültige Lösungen für übliche Probleme.
Sie wurden empirisch beobachtet und gelten daher meistens, allerdings auch nicht unbedingt immer.
(Vgl.~\cite{getting-a-solid-start})

\subsubsection{SOLID Principles}
Die Kerngruppe der Software Design Prinzipien werden von Robert C. Martin unter dem Titel \enquote{The Principles of OOD}\cite{solid} beschrieben.
Die Grundlage dafür legte er in seinem Paper \enquote{Design Principles and Design Patterns}\cite{design-principles-and-design-patterns} im Jahr 2000.

\newlineparagraph{Single Responsibility Principle}
\enquote{A class should have one, and only one, reason to change.}\cite{solid}
Eine Klasse sollte immer so designt werden, dass sie nur einen Zweck erfüllt\cite{different-types-of-software-design-principles}.

\color{red}
Ziel definieren im Projektkontext
\color{black}

\newlineparagraph{Open Closed Principle}
\enquote{You should be able to extend a classes' behavior, without modifying it.}\cite{solid}
Fertige Klassen sollten nicht mehr verändert werden, lediglich Bugs sollten behoben werden.
Allerdings sollte es immer möglich sein, die Funktionalität zu erweitern.
Dafür ist es wichtig, dass die Klassen so entworfen sind, dass Unterklassen die Funktionalität durch Overwrites nicht einschränken können.
Dafür müssen vor allem relevante Typen definiert werden und wichtige und finale Methoden auch als solche markiert werden, damit diese nicht mehr verändert werden können.
(Vgl.~\cite{different-types-of-software-design-principles})

\color{red}
Ziel definieren im Projektkontext
\color{black}

\newpage

\newlineparagraph{Liskov Substitution Principle}
\enquote{Derived classes must be substitutable for their base classes.}\cite{solid}
Eine Unterklasse sollte immer genau wie die Oberklasse verwendet werden können, die sie erweitert\cite{different-types-of-software-design-principles}.

\color{red}
Ziel definieren im Projektkontext
\color{black}

\newlineparagraph{Interface Segregation Principle}
\enquote{Make fine-grained interfaces that are client specific.}\cite{solid}
Interfaces sollten genau wie Klassen möglichst spezifisch sein.
Ansonsten würden die implementierenden Klassen gegen das Single Responsibility Principle verstoßen.
(Vgl.~\cite{design-patterns-php-laravel})

\color{red}
Ziel definieren im Projektkontext
\color{black}

\newlineparagraph{Dependency Inversion Principle}
\enquote{Depend on abstractions, not on concretions.}\cite{solid}
Für eine möglichst lose Kopplung zwischen konkreten Klassen sollte man sich auf Interfaces und abstrakte Klassen verlassen.
So wird eine flexiblere Codebase ermöglicht.
(Vgl.~\cite{design-patterns-php-laravel})

\color{red}
Ziel definieren im Projektkontext
\color{black}

\newpage

\subsubsection{KISS (Keep It Simple Stupid)}
Das KISS Prinzip wurde ursprünglich im amerikanischen Militär geprägt und ist zurückzuführen auf Kelly Johnson.
(Vgl.~\cite{kelly-johnson-memoir})

Im Bereich Software Design sollen Systeme möglichst einfach, bestehend aus möglichst wenigen Teilen, gebaut werden.
Ebenso sollen Abhängigkeiten und interne Verbindungen reduziert werden.
Außerdem soll die Codebasis so gestaltet werden, dass andere Entwickler, auch Anfänger, sie verändern können.
(Vgl.~\cite{kiss-principle-explained})

\subsubsection{DRY (Don’t Repeat Yourself)}
Denselben Code immer und immer wieder zu wiederholen ist nicht erstrebenswert.
Wenn Wiederholungen bemerkt werden, sollten sie immer in Methoden oder andere geeignete Strukturen ausgelagert werden.
Dadurch wird zum einen der Code insgesamt kürzer, zum anderen werden Wartungen einfacher.
(Vgl.~\cite{the-pragmatic-programmer})

Beim Einsatz von Frameworks ist zu beachten, dass diese oft an mehreren Stellen dieselben Konfigurationen benötigen.
Wenn zum Beispiel bei zu erstellenden Klassen ein oder mehrere Attribute auf denselben Wert gesetzt werden sollen, dann bietet es sich an, eine Oberklasse zu erstellen die diese Konfiguration bereits setzt.

\subsubsection{YAGNI (You aren't gonna need it)}
Eine Implementierung sollte immer nur dann erfolgen, wenn sie konkret notwendig ist.
Wenn lediglich vorhersehbar ist, dass eine gewisse Funktion notwendig sein wird, dann sollte sie noch nicht implementiert werden.
(Vgl.~\cite{extreme-programming-installed})

Dieser Grundsatz wurde im Bereich des Extreme Programming geprägt.
Begründet sieht er sich darin, dass es sich nur selten lohnt, für zukünftige Anforderungen zu entwickeln.
(Vgl.~\cite{kiss-principle-explained})

\subsubsection{PSR (PHP Standards Recommendations)}
Die PHP FIG (Framework Interop Group) arbeitet an und veröffentlicht PSRs (PHP Standards Recommendations).
Eine moderne IDE (Integrierte Entwicklungsumgebung) prüft PHP Code meist auf die PSR-1 (Basic Coding Standard) und vor allem die PSR-12 (Extended Coding Style).
Die PHP-FIG regelt auch weitere Standards, wie zum Beispiel Logger Interfaces, Autoloading und HTTP Features.
(Vgl.~\cite{psr})
