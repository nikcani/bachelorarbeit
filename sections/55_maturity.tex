\subsection{Maturity im Vergleich}
Unter Anwendung des TRL (Technology Readiness Level) ließen sich keine Unterschiede zwischen Nova und filament feststellen.
Im Rahmen dieser Arbeit würden beide Umsetzungen Level 7 erreichen, da eine prototypische Demo in einer operativen Umgebung fertiggestellt ist.
Werden jedoch die Berichte aus der Community als Basis zugrunde gelegt, so müssen beide Frameworks auf Level 9, dem höchsten Level, eingeordnet werden.
Sowohl für Nova als auch für filament laufen erfolgreiche Anwendungen im Produktivbetrieb.
(Vgl.~\cite{technology-readiness})

Eine stabile Version von Nova wurde erstmals am 22.08.2018\cite{nova-releases} veröffentlicht und ist damit etwa viereinhalb Jahre alt.
Filament ist wesentlich jünger, das erste stable Release erfolgte am 02.03.2021\cite{filament-releases}, also vor knapp zwei Jahren.

Nova ist folglich mehr als doppelt so lange verfügbar wie filament und befindet sich inzwischen in der vierten Hauptversion\cite{nova-releases}.
Filament hat bisher die zweite Hauptversion\cite{filament-releases} erreicht.
Wird das Alter der Technologie als Indikator für die Reife herangezogen, so spricht dieses für Nova.

\newlineparagraph{Dokumentation}
Bei der praktischen Arbeit mit beiden Frameworks ist aufgefallen, dass die Dokumentation in beiden Fällen hervorragend ist.
Es war nie notwendig das Framework selbstständig nachzuvollziehen.
Alle relevanten Features sind ausführlich und an Beispielen beschrieben.
Auf Basis der Dokumentation lässt sich daher kein Favorit ermitteln, beide Frameworks schneiden gleich gut ab.

\newlineparagraph{Nutzerzahl}
Bei filament lassen sich konkrete Zahlen ermitteln (Stand: 08.01.2023 11:30 Uhr).
Packagist verzeichnet insgesamt 470.626 Installationen.
Bei GitHub gibt es 5.4k Stars, 795 Forks und 91 Watcher.

Bei Nova hingegen lassen sich keine konkreten Zahlen ermitteln.
Es gibt lediglich ein offenes GitHub Projekt in dem Fehler gesammelt werden.
Dieses hat 525 Stars, 36 Forks und 58 Watcher.
Da diese Zahlen nicht vergleichbar sind, ist auf Basis der Nutzerzahlen keine Aussage möglich.

\newlineparagraph{Bekannte Fehler}
Nova listet aktuell 12 offene Issues, bei filament sind es 18, davon 7 für die kommende Version.
Der quantitative Blick auf die Fehlerdokumentationen lässt daher keinen Unterschied feststellen.

\newlineparagraph{Moderne Sprachfeatures}
Beide Frameworks laufen unter PHP 8.2 und damit unter der aktuell neusten Version.
Moderne Sprachfeatures können daher bei beiden Frameworks verwendet werden, hier ist kein Unterschied zu erkennen.

\newlineparagraph{Roadmap}
Bei Nova gibt es keine öffentliche Roadmap, entsprechend ist hier kein Vergleich möglich.

Bei filament ist seit mehreren Monaten die nächste Hauptversion (v3) in Entwicklung\cite{filament-v3-plans} und die Pläne werden offen kommuniziert.
Das Framework soll mit der nächsten Version nicht mehr nur für Admin Panels, sondern für jegliche Anwendung konzipiert werden\cite{filament-v3}.
Dies ist für das vorliegende Projekt der Sounding Console sehr interessant, da es hier bereits über ein Admin Panel hinaus verwendet wird.

Filament überzeugt basierend auf der Roadmap gegenüber Nova.
Die geplanten Features sind für den aktuellen Projektkontext sehr interessant.
Dazu gehören insbesondere der Support genereller Anwendungen, verbesserte Performance, visuell ansprechendere Views und zusätzliche Icons.

\newlineparagraph{Fazit Maturity}
Nova überzeugt durch ein wesentlich höheres Alter, filament durch die verfügbare Roadmap.
An dieser Stelle lässt sich hinsichtlich der Maturity kein Unterschied zwischen den beiden Frameworks feststellen.
