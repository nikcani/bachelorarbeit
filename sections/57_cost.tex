\subsection{Kostenvergleich/Lizenzierung}
Die Kosten beim Einsatz von Nova und filament unterscheiden sich deutlich.
Filament ist eine quelloffene Entwicklung unter der MIT-Lizenz.
Dadurch sind die vollständige Nutzung, Veränderungen und der Vertrieb möglich.
Es entstehen keine Kosten und eine Weiterentwicklung wäre auch durch neue Entwickler möglich, sollten die aktuellen Entwickler das Projekt einstellen.

Nova hingegen kostet für unlimitierte Projekte einmalig 299 Dollar.
Durch die On-Premise Bereitstellung ist diese größere Lizenz notwendig.
Diese Lizenz ermöglicht Updates für ein Jahr, danach fallen erneut Kosten an.
Da Updates notwendig sind, ist mit jährlich wiederkehrenden Kosten zu rechnen.
Gleichwohl fallen bei beiden Frameworks Personalkosten an.
Mit Blick hierauf sind die Lizenzkosten vergleichsweise niedrig.

Bei der Betrachtung der Lizenzkosten schneidet filament somit besser ab.
Neben den fehlenden Lizenzkosten ist vor allem die quelloffene Entwicklung von Vorteil, insbesondere hinsichtlich langer Produktlebenszyklen.
