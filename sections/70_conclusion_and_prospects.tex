\newpage


\section{Fazit und Ausblick}

\subsection{Fazit}
Die Durchführung des Vergleichs ist für das Projekt und das daraus resultierende Produkt von hohem Nutzen.
Es war möglich auf der bisherigen Entwicklung aufzubauen und die UI auszutauschen.
Individuelle Komponenten konnten wiederverwendet und so auch der zeitliche Rahmen eingehalten werden.

Dabei konnte filament klar gegenüber Nova überzeugen.
Es bietet viele Funktionen, die Anpassbarkeit ist bereits heute sehr gut.
Zudem sind, wie dargestellt, sogar Verbesserungen für die künftige Entwicklung avisiert.
Die neue technische Basis für die Sounding Console steht somit fest.

\subsection{Ausblick}
Auf der Basis des eindeutigen Fazits zugunsten des Einsatzes von filament soll die Sounding Console zeitnah vollständig auf filament umgestellt werden.
Dafür müssen die wenigen, gegenüber dem MVC mit Nova fehlenden, Features implementiert werden.

Im Anschluss daran soll eine erste interne Testphase beginnen und die Software von mehreren Nutzern evaluiert werden.
Nach Anpassungen, die sich aus dieser Evaluation ergeben, soll die Software bei den ersten Messnetzwerken in Betrieb gehen.

Bisher wurden realistische Lasttests noch nicht durchgeführt.
Diese wurden bereits im Ausblick der Praxisprojektarbeit angekündigt und die Bearbeitung ist bereits gestartet.
Jedoch wurde zunächst der Vergleich mit, sowie der Umbau auf filament priorisiert.
Da zudem bei den Lasttests kein Browser ausgeführt wird, sondern nur HTTP Endpunkte abgerufen werden, ist das Testing von filament voraussichtlich besser zu bewerkstelligen.
Nova setzt viel auf Code im Frontend und führt auch dort das Routing durch.
Filament setzt die meisten Funktionen rein im Backend um und lässt sich insofern mit der angestrebten Testmethode auch besser testen, als dies bei Nova der Fall wäre.

Sobald sich die Software im produktiven Betrieb bewiesen hat, sollen weitere Funktionen, zum Beispiel für automatisierte Bodenstationen, sogenannte Autolauncher, implementiert werden.
Hierzu liegen bereits erste grobe Pläne vor.
