\section{Grundlagen}
Die theoretische Basis für den Vergleich bieten verschiedene Software Design Principles und Patterns, welche
im Folgenden eingeführt werden.

\subsection{Principles}
Software Design Principles sind weder Regeln, Gesetze noch perfekte Wahrheiten.
Stattdessen sind diese Prinzipien Empfehlungen bzw.\ Ratschläge.
(Vgl.~\cite{getting-a-solid-start})

Software Design Prinzipien sind Heuristiken und allgemeingültige Lösungen für übliche Probleme.
Sie wurden empirisch beobachtet und gelten daher meistens, allerdings auch nicht unbedingt immer.
(Vgl.~\cite{getting-a-solid-start})

https://www.geeksforgeeks.org/principles-of-software-design/

https://www.dotnettricks.com/learn/designpatterns/different-types-of-software-design-principles

\subsubsection{SOLID Principles}
Die Kerngruppe der Software Design Prinzipien werden von Robert C. Martin unter dem Titel \enquote{The Principles of OOD}\cite{solid} beschrieben.
Die Grundlage dafür legte er in seinem Paper \enquote{Design Principles and Design Patterns}\cite{design-principles-and-design-patterns} im Jahr 2000.

https://medium.com/successivetech/s-o-l-i-d-the-first-5-principles-of-object-oriented-design-with-php-b6d2742c90d7

https://www.digitalocean.com/community/conceptual-articles/s-o-l-i-d-the-first-five-principles-of-object-oriented-design#interface-segregation-principle

https://accesto.com/blog/solid-php-solid-principles-in-php/

\paragraph{Single Responsibility Principle}
\enquote{A class should have one, and only one, reason to change.}\cite{solid}

\paragraph{Open Closed Principle}
\enquote{You should be able to extend a classes' behavior, without modifying it.}\cite{solid}

\paragraph{Liskov Substitution Principle}
\enquote{Derived classes must be substitutable for their base classes.}\cite{solid}

\paragraph{Interface Segregation Principle}
\enquote{Make fine-grained interfaces that are client specific.}\cite{solid}

\paragraph{Dependency Inversion Principle}
\enquote{Depend on abstractions, not on concretions.}\cite{solid}

\subsubsection{Boy Scout Rule}
Sinnvoll im Projektkontext?

\subsubsection{PSR (PHP Standards Recommendations)}
Die PHP FIG (Framework Interop Group) arbeitet an und veröffentlicht PSRs (PHP Standards Recommendations).
Eine moderne IDE (Integrierte Entwicklungsumgebung) prüft PHP Code meist auf die PSR-1 (Basic Coding Standard) und vor allem die PSR-12 (Extended Coding Style Guide).
Die PHP-FIG regelt auch weitere Standards, wie zum Beispiel Logger Interfaces, Autoloading und HTTP Features.
(Vgl.~\cite{psr})

\subsubsection{KISS (Keep It Simple Stupid)}
Das KISS Prinzip wurde ursprünglich im amerikanischen Militär geprägt und ist zurückzuführen auf Kelly Johnson.
(Vgl.~\cite{kelly-johnson-memoir})

Im Bereich Software Design sollen Systeme möglichst einfach, bestehend aus möglichst wenigen Teilen, gebaut werden.
Ebenso sollen Abhängigkeiten und interne Verbindungen reduziert werden.
Außerdem soll die Codebasis so gestaltet werden, dass andere Entwickler, auch Anfänger, sie verändern können.
(Vgl.~\cite{kiss-principle-explained})

\subsubsection{DRY (Don’t Repeat Yourself)}
Den selben Code immer und immer wieder zu wiederholen ist nicht erstrebenswert.
Wenn Wiederholungen bemerkt werden, sollten sie immer in Methoden oder andere geeignete Strukturen ausgelagert werden.
Dadurch wird zum einen der Code insgesamt kürzer, zum anderen werden Wartungen einfacher.
(Vgl.~\cite{the-pragmatic-programmer})

Beim Einsatz von Frameworks ist zu beachten, dass diese oft an mehreren Stellen dieselben Konfigurationen benötigen.
Wenn zum Beispiel bei zu erstellenden Klassen ein oder mehrere Attribute auf denselben Wert gesetzt werden sollen, dann bietet es sich an, eine Oberklasse zu erstellen die diese Konfiguration bereits setzt.

\subsubsection{YAGNI (You aren't gonna need it)}
Eine Implementierung sollte immer nur dann erfolgen, wenn sie konkret notwendig ist.
Wenn lediglich vorhersehbar ist, dass eine gewisse Funktion notwendig sein wird, dann sollte sie noch nicht implementiert werden.
(Vgl.~\cite{extreme-programming-installed})

Dieser Grundsatz wurde im Bereich des Extreme Programming geprägt.
Begründet sieht er sich darin, dass es sich nur selten lohnt, für zukünftige Anforderungen zu entwickeln.
(Vgl.~\cite{kiss-principle-explained})

\subsection{Patterns}
Patterns, sogenannte Entwurfsmuster, wurden ursprünglich vom Architekten Christopher Alexander geprägt.
Dieser beschrieb in seinem Buch \enquote{A Pattern Language: Towns, Buildings, Construction}\cite{a-pattern-language} im Jahr 1977 zum ersten Mal Muster, unter dessen Einsatz man immer wiederkehrende Probleme lösen kann.
In der Informatik wurden Entwurfsmuster durch die Veröffentlichung der \enquote{Gang of Four} im Jahr 1994 populärer.
In ihrem Buch \enquote{Design Patterns - Elements of Reusable Object-Oriented Software}\cite{gamma-design-patterns} beschreiben Erich Gamma et al.\ 23 unterschiedliche Entwurfsmuster.
Diese sind eingeteilt in die drei Kategorien Creational, Structural und Behavioral.

1999 ergänzte Martin Fowler in \enquote{Patterns of Enterprise Application Architecture}\cite{patterns-of-enterprise-application-architecture} die Kategorie \enquote{Objektrelationale Abbildung} und dazugehörige Muster.
Gregor Hohpe und Bobby Woolf ergänzten 2003 die Kategorie der Messaging Patterns in ihrem Buch \enquote{Enterprise Integration Patterns}\cite{enterprise-integration-patterns}.

Die Menge der Entwurfsmuster ist so umfangreich, dass eine wiedergabe aller an dieser Stelle weder von Nutzen noch machbar wäre.
Im Folgenden werden daher nur die Muster eingeführt, die im anschließenden Vergleich verwendet werden.

\subsubsection{GRASP (General Responsibility Assignment Software Patterns)}
Sinnvoll im Projektkontext?

GRASP is a large set of rules about which I could write a separate article.
These are the basic principles that we should follow when creating object design and responsibility assignments.
It consists of: Information Expert, Controller, Creator, High Cohesion, Low Coupling, Pure Fabrication, Polymorphism, Protected Variations, Indirection.
