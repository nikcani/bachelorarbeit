\section{Analyse}
Die Basis für den praktischen Teil dieser Arbeit ist die auf Basis von Laravel und Nova entwickelte Sounding Console.
Um den Funktionsrahmen für den weiteren Prototypen zu definieren, sind zunächst die Probleme mit dem bestehenden MVP (Minimal Viable Product) zu identifizieren.

Die Basis für den theoretischen Teil liefern die skizzierten Software Design Principles und Patterns.
Im Rahmen der weiteren Analyse wird geklärt, welche Principles und Patterns im vorliegenden technischen Anwendungskontext grundsätzlich zur Anwendung kommen können.

Weiterhin bietet der jeweilige Funktionsumfang der Frameworks eine sinnvolle Vergleichsbasis.
Dafür wird eine geeignete Vergleichsmethodik identifiziert, definiert und durchgeführt.

Weitere mögliche Vergleichsaspekte sind die Maturity der beiden Frameworks, der verwendete Code Style, ein Kostenvergleich und Unterschiede in der Lizenzierung.
Inwiefern diese Aspekte vergleichbar sind, ist ebenfalls Gegenstand dieser Analyse.

\subsection{Probleme des bisherigen Prototyps}
Im Rahmen der Nutzwertanalyse wurde angenommen, dass filament im Vergleich zu Nova besser anpassbar ist.
Insbesondere die Flight View wurde in Tests von Anwendern kritisiert.
Nova zeigte bezogen auf die Platzierung von individuellen Komponenten innerhalb einer View wenig Spielraum.
Im praktischen Prototyp soll nun geprüft werden, ob filament in diesem Fall bessere Anpassungsmöglichkeiten bietet.
Zudem ergaben sich die folgenden Probleme bei der Umsetzung des MVP mit Nova.

\newlineparagraph{Styling von Custom Components}
Trotz individueller Komponenten wird ein einheitliche UI (User Interface) angestrebt.
Dazu werden innerhalb der individuellen Komponenten UI-Components von Nova importiert.
Dies ist bei Nova jedoch nicht vorgesehen.
Es gilt daher zu klären, ob sich dies bei filament anders darstellt.

\newlineparagraph{Infinite Loading Tables}
Die unschön gelösten Infinite Loading Tabellen in Nova sind problematisch, da die Funktion mit einem Update jederzeit unbrauchbar gemacht werden könnte.
Filament bietet in der Pagination die Möglichkeit, alle Datensätze in einer Tabelle zu laden.
Es wird daher geprüft, ob die Performance hierzu ausreichend ist oder ob es eine andere Lösung gibt.

\newpage

\newlineparagraph{Tabellensortierung}
Die fehlende Möglichkeit, bei der Messwertetabelle eine Standardsortierung zu definieren, ist ein weiterer Kritikpunkt bei Nova.
Beim Einsatz von filament wird geprüft, ob dieses Problem gelöst werden kann.

\newlineparagraph{Polling Tables}
Manche Tabellen, beispielsweise die Stationsübersicht, sollen in einem gewissen Intervall automatisch abgerufen werden.
So können Anwender stets erkennen, welche Stationen aktuell online sind und welche nicht.
Nova bietet zwar diese Funktionalität, allerdings ist die Animation sehr störend:
Die Tabelle verschwindet für einen Moment und dadurch ist ein einfacher und schneller Vergleich schwierig.
Außerdem wird die Tabelle kürzer während der Animation.
Dadurch springt die ganze Seite und erschwert so den Fokus des Users.

Filament bietet das notwendige Feature ebenfalls.
Es wird im Prototyp daher geprüft, ob die Animation besser umgesetzt ist.

\subsection{Geeignete Principles und Patterns}
Die Auswahl der Principles und Patterns im Grundlagenteil dieser Arbeit erfolgte bereits begründet.
Nicht jedes Pattern passt in jedem Kontext.
Im Folgenden wird begründet, warum die skizzierten Principles und Patterns im vorliegenden Projektkontext überprüft werden.

\newlineparagraph{Geeignete Principles}
Die Sounding Console folgt dem Konzept der objektorientierten Entwicklung.
Dafür sind die SOLID Principles entworfen und daher im vorliegenden Projekt geeignet.

Die Prinzipien KISS, DRY und YAGNI sind generell für unterschiedlichste Softwareprojekte geeignet.
Daher wird der Einsatz der Frameworks untersucht auf Einfachheit, Wiederholungen und vorbereitenden, aber nicht notwendigen Code.

Filament und Nova sind, abgesehen vom Frontend, in PHP geschrieben.
Beim Einsatz der Frameworks wird fast ausschließlich PHP verwendet.
Die PSR ist daher der geeignete Coding Standard, der hier zur Anwendung kommt.

\newlineparagraph{Geeignete Patterns}
Im Bereich der Webentwicklung kommt meist das MVC Pattern zum Einsatz.
Es ist daher zu prüfen, ob dies der Fall ist oder ob PAC verwendet wird.
Zusätzlich ist zu erkunden, ob das View Handler Pattern zum Einsatz kommt.

Die Sounding Console besteht aus mehreren Komponenten.
Es könnten daher einige als Whole-Part organisiert oder in Layer aufgeteilt sein, dies ist zu prüfen.
Ebenfalls anwendbar für die Kommunikation zwischen Komponenten wäre das Publisher-Subscriber Pattern.
Ob die Frameworks dieses Pattern ermöglichen, ist zu überprüfen.

Weitere mögliche Patterns im Bereich der Kommunikation zwischen Komponenten sind ein Broker sowie ein Proxy.
Es ist zu kontrollieren, ob diese Patterns unterstützt werden und zur Anwendung kommen.
